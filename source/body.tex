\section*{Question \#1}


\question{Consider the angle $\theta$ in with measure $-495^\circ$.}

The problem with this question is that I was trying to be smart by simplifying $-495^\circ$ to $225^\circ$, so thereafter all of my subsequent work was using an angle co-terminal to $-495^\circ$. In this context, my mistake was equating co-terminal angles as being the same, but this only applies to the output of periodic functions where co-terminal angles \textbf{map to the same value}, but the arguments themselves represent different measures.

\subsection*{Question \#1 (A)}

Using the following relation:
\begin{equation}
\begin{split}
x^{\circ} = \frac{x}{360} \tau \; {\displaystyle {\text{rad}}}
\end{split}
\end{equation}

I can express $-495^\circ$ in terms of turns (or $\tau$) like so:
\begin{equation}
\begin{split}
-495^{\circ} &= \frac{-495}{360} \tau \; {\displaystyle {\text{rad}}} \\
    &= -\frac{11}{8} \tau \; {\displaystyle {\text{rad}}} \\
    &= - 1\frac{3}{8} \; \tau
\end{split}
\end{equation}

Since this angle is expressed in terms of turns, we have a intuitive idea about how the graph the given angle. I.e. because $-\frac{3}{8}$ can be considered a ratio of a circle, which is easy to picture, in the same manner that $\frac{3}{4}$ of a circle is easy to imagine, compared to e.g. $\frac{3}{2}$ half circles.

\newpage

Therefore, I know that this angle makes one full revolution, and $-\frac{3}{8}$ of a revolution (going clockwise), which results in the following figure correctly drawn in the context of a $-495^\circ$ angle:

\begin{figure}[!h]
\centering
\begin{tikzpicture}
\tikzset{>=stealth}
% draw axises and labels. We store a single coordinate to have the
% direction of the x axis
\draw[->] (-4,0) -- ++(8,0) coordinate (X) node[below] {$x$};
\draw[->] (0,-4) -- ++(0,8) node[left] {$y$};

\newcommand\CircleRadius{3cm}
\draw (0,0) circle (\CircleRadius);
% special method of noting the position of a point
\coordinate (P) at (-495:\CircleRadius);

\draw[thick] 
(0,0)
coordinate (O) % store origin
node[] {} % label
-- 
node[below left, pos=1] {$P(-\frac{\sqrt{2}}{2}, -\frac{\sqrt{2}}{2})$} % some labels
node[below right, midway] {$r$}
(P) 
-- 
node[midway,left] {$y$}
(P |- O) coordinate (Px) % projection onto horizontal line through
                            % O, saved for later
-- 
node[midway, below] {$x$}
cycle % closed path

% pic trick is from the angles library, requires the three points of
% the marked angle to be named

pic [] {angle=X--O--P};
\draw[->,red] (5mm, 0mm) arc (0:-495:5mm) node[midway,xshift=-4mm,yshift=3.5mm] {$-495^\circ$};
% right angle marker
\draw ($(Px)+(0.3, 0)$) -- ++(0, -0.3) -- ++(-0.3,0);
\end{tikzpicture}
\end{figure}

\subsection*{Question \#1 (B)}

Using the following relation:
\begin{equation}
\begin{split}
x^{\circ} = \frac{x}{360} (2\pi) \; {\displaystyle {\text{rad}}}
\end{split}
\end{equation}

I can \textbf{correctly} express $-495^\circ$ degrees in terms of radians like so:
\begin{equation}
\begin{split}
-495^{\circ} &= \frac{-495}{360} (2\pi) \; {\displaystyle {\text{rad}}} \\
    &= -\frac{11}{8} (2\pi) \; {\displaystyle {\text{rad}}} \\
    &= -\frac{11}{4} \pi \; {\displaystyle {\text{rad}}}
\end{split}
\end{equation}

\answer{$-\frac{11}{4} \pi \; {\displaystyle {\text{rad}}}$}


\subsection*{Question \#1 (C)}

\section*{Question \#3}

This should have been correct, what happens is my brain frequently gets things mixed up, i.e. there was a disconnect between my internal thoughts and what manifested on paper. I know that $\sec(x)$ is the reciprocal of $\cos(x)$, just as $\csc(x)$ is the reciprocal of $\sin(x)$, and that $\cot(x)$ is the reciprocal of $\tan(x)$, and furthermore in the context of the unit circle, that $\sin(\theta)$ represents the y value, and that $\cos(\theta)$ represents the x value. 

Therefore, given the some $P(-20, 21)$ for $\theta$:

\begin{equation}
\begin{split}
r &= \sqrt{x^2 + y^2} = 29 \\
\sin(\theta) &= \frac{y}{r} = \frac{21}{29} \\
\sec(\theta) &= \frac{r}{x} = \frac{29}{-20} = -\frac{29}{20}
\end{split}
\end{equation}

\answer{$\sin(\theta) = \frac{21}{29}$, $\sec(\theta) = -\frac{29}{20}$}
    

\newpage
\section*{Question \#4}

% \begin{figure}[!h]
% \centering
% \begin{tikzpicture}[scale=1.25]
% \coordinate [label=left:$F$] (A) at (-1.5cm,-1.cm);
% \coordinate [label=right:$G$] (C) at (1.5cm,-1.0cm);
% \coordinate [label=above:$E$] (B) at (1.5cm,1.0cm);
% \draw 
%     (A) --
% node[above] {$a$} (B) --
% node[right] {$c$} (C) -- 
% node[below] {$b$} (A);
% \draw
% (1.25cm,-1.0cm) rectangle (1.5cm,-0.75cm);
% \tkzMarkAngle[size=1cm,color=cyan,mark=|](C,A,B);
% \end{tikzpicture}
% \end{figure}


\begin{figure}[!h]
\centering
\begin{tikzpicture}

    \coordinate [label=below:$F$] (F) at (0,0);
    \coordinate [label=below:$G$] (G) at (2,0);
    \coordinate [label=below:$H$] (H) at (5,0);
    \coordinate [label=above:$E$] (E) at (5,5);

    \draw (F)--(E);
    \draw (F)--(H);
    \draw (H)--(E);
    \draw (E)--(G);
\end{tikzpicture}
\end{figure}

Information given:

\begin{itemize}
    \item $\overline{FG} = 600  {\displaystyle {\text{m}}}$
    \item $\measuredangle EFH = 1.91^\circ$
    \item $\measuredangle EGH = 2.67^\circ$
\end{itemize}

During the exam I was stuck on the gap between $G$ and $H$, in hindsight it all makes sense, i.e. just solve for $\overline{GH}$ using an unknown value for $\overline{EH}$, since we can factor this quantity out in the ensuing expression for $\tan(1.91^\circ)$. Although this realization occurred after spending an hour or so on the problem, I suppose failing this question was inevitable.

Solution:

\begin{equation}
\begin{split}
\tan(2.67^\circ) &= \frac{\overline{EH}}{x} \\
x &= \frac{\overline{EH}}{\tan(2.67^\circ)} \\
\tan(1.91^\circ) &= \frac{\overline{EH}}{600 + x} \\
(600 + x)(\tan(1.91^\circ)) &= (600 + x)\frac{\overline{EH}}{600 + x} \\
(600 + x)(\tan(1.91^\circ)) &= \overline{EH} \\
600 \cdot \tan(1.91^\circ) + x \cdot \tan(1.91^\circ) &= \overline{EH} \\
600 \cdot \tan(1.91^\circ) &= \overline{EH} - \overline{EH} \frac{\tan(1.91^\circ)}{\tan(2.67^\circ)} \\
600 \cdot \tan(1.91^\circ) &= \overline{EH}(1 - \frac{\tan(1.91^\circ)}{\tan(2.67^\circ)}) \\
\frac{600 \cdot \tan(1.91^\circ)}{1 - \frac{\tan(1.91^\circ)}{\tan(2.67^\circ)}} &= \overline{EH} \approx 70
\end{split}
\end{equation}



Why did it not occur to me to use $600 + x$? After thinking about it, I was time constrained, so instead of working though something without certainty, I was trying to find predefined solutions to predefined problems. Whereas if I started incrementally, $A$ may have lead to $B$, which may have lead to an obvious answer.


\answer{$\overline{EH} \approx 70 \; {\displaystyle {\text{m}}} $}





\section*{Question \#5}

\section*{Question \#5 (A)}

Again, this should have been correct, I made a dumb arithmetic error, I was rushing through my brain short circuited. Anyway the \textbf{correct} period for this function is:
\begin{equation}
\begin{split}
\frac{\pi}{\frac{1}{3}} = \pi \cdot \frac{3}{1} = 3\pi
\end{split}
\end{equation}

\answer{$3\pi$}

\newpage
\section*{Question \#5 (B)}

By default, the period of the tangent function is $\pi$, and it's easiest to graph a single period within the asymptotes, such as from $-\frac{1}{2}\pi$ to $\frac{1}{2}\pi$, which is what I did, but didn't update the labels with the halved period, which is definitely an error. 

\begin{tikzpicture}[domain=-4:7] [scale=0.8]
\draw[ultra thick, ->] (-4,0) -- (7.5,0) node[right] {$\textbf{X}$};
\draw[ultra thick,->] (0,-2) -- (0,4.2) node[above] {$\textbf{Y}$};
\draw[dashed, thick,red] (1.57,0) -- (1.57,1);
\draw[dashed, thick,red] (4.71,0) -- (4.71,1);
\draw[dashed, thick,red] (-1.57,0) -- (-1.57,1);
\draw[ultra thick,color=blue] plot[domain=-.4*pi:.4*pi] (\x,{-tan(\x r)}) node[right] {$y=-\frac{1}{2}\cdot\tan(\frac{1}{3} \cdot x)$};

\coordinate [label=below:$\frac{3}{2}\pi$] (E) at (1.57,0);
\coordinate [label=below:$-\frac{3}{2}\pi$] (E) at (-1.57,0);
\end{tikzpicture}



\section*{Question \#6}

This question should have been perfect, but my brain short circuited with the period, again. Anyway the \textbf{correct} period for this function is:
\begin{equation}
\begin{split}
\frac{2\pi}{\frac{1}{3}} = 2\pi \cdot \frac{3}{1} = 6\pi
\end{split}
\end{equation}


\section*{Question \#7}

It's weird that I correctly computed the period for this function (which was more difficult), I even recomputed such in the same manner as the above two problems, yet without error. How does this happen? It's like I thought $2+2 = 5$, it makes no sense... 


\begin{equation}
\begin{split}
\frac{2\pi}{\frac{1}{2}} = 2\pi \cdot \frac{2}{1} = 2\pi
\end{split}
\end{equation}



% \begin{enumerate}
%     \item[1.]
%     \begin{enumerate}
%         \item[(a)] One entry in the list
%         \item[(b)] Another entry in the list
%     \end{enumerate}
%     \item[4.] Another entry in the list
% \end{enumerate}
